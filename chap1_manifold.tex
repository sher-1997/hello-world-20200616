\chapter{预备知识}
	\noindent
	%\hangafter=1\hangindent 3.5em
	{\textbf{1.1\ 证:}}令$V=\mathbb{R}^n,\  A=\mathbb{R}^n,\ $ 则\  $\forall\,x,\ y\in{A}$,\ 令\ $\overrightarrow{xy}=y-x\,$则有:
	\begin{itemize}
		\item[\textcircled{1}]w
		$\forall\ x\in{A},\ \overrightarrow{xx}=x-x=\mathit{0}$,
		\item[\textcircled{2}]
		$\forall\ x\in{A},\ v\in{V},\ $令\ $y=x+v\in{V}$.\ 则\ $\overrightarrow{xy}=y-x=v$.\ 若另有\ $y',\ s.\ t.\ \overrightarrow{xy'}=v$,\ 则\ $y'-x=v\Rightarrow y'=x+v=y$\ 从而\ y\ 唯一,
		\item[\textcircled{3}]
		$\forall\ x,y,z\in A.\ \overrightarrow{xy}+\overrightarrow{yz}=(y-x)+(z-y)=z-y=\overrightarrow{yz}.\ $
	\end{itemize}
	所以 \ $\mathbb{R}^n$\ 是一个n维仿射空间,它以 \ $\mathbb{R}^n$\ 自身为它的伴随向量空间。
	\\
	\\
	
	\noindent
	{\textbf{1.2\ 证:}}
	\begin{itemize}
		\item[\textcircled{1}]
		$\forall\  P,Q\in E^n,\ 0\leqslant d(P,Q)=\left|\overrightarrow{PQ}\right|<+\infty,$\ 且$d(P,Q)=0\Leftrightarrow\left|\overrightarrow{PQ}\right|=0\Leftrightarrow P=Q,$ 
		\item[\textcircled{2}] 
		$\forall\ P,Q\in E^n,\ d(P,Q)=\left|\overrightarrow{PQ}\right|=\left|\overrightarrow{QP}\right|=d(Q,P),$
		\item[\textcircled{3}] 
		$\forall\  P, Q, R\in E^n,\  d(P,R)=\left|\overrightarrow{PR}\right|=\left|\overrightarrow{PQ}+\overrightarrow{QR}\right|\leqslant\left|\overrightarrow{PQ}\right|+\left|\overrightarrow{QR}\right|\leqslant d(Q,P).$
	\end{itemize}
	从而\ $E^n$\ 关于距离函数\ $d$\ 成为一个度量。
	\\
	\\
	
	
	\noindent
	{\textbf{1.3\ 证:}}
	\begin{itemize}
		\item[(1)] 记\ $E^n$\ 的全体开子集为\ $\tau$,
		\begin{itemize}
			\item[\textcircled{1}] 显然\ $\emptyset,\ E^n\in\tau,$
			\item[\textcircled{2}] $\forall\ A\in{\tau},\ B\in \tau ,\ $
			若\ $A\cap B=\emptyset,\ $则\ $A\cap B\in \tau,\ $
			\\
			若\ $A\cap B\neq \emptyset,\ $则\ $\forall\ P\in A\cap B,\ $即\ $P\in A$\ 且\ $P\in B,\ $
			\\
			则\ $\exists\ \varepsilon_1,\ \varepsilon_2 >0,\ s.\ t.\ P\in B_{\varepsilon_1} (P)\subset A,\ P\in B_{\varepsilon_2} (P) \subset B $,\ 取\ $\varepsilon=min {\left\{\varepsilon_1,\varepsilon_2\right\}}$,\ 
			则\ $P\in
			B_\varepsilon(P)=B_{\varepsilon_1}(P)\cap B_{\varepsilon_2}(P)\subset A\cap B,\ $因而\ $A\cap B\in \tau.$
			\item[\textcircled{3}] 若\ $A_\alpha (\alpha\in I)\in \tau, \ $则\ $\forall\ P\in \bigcup\limits_{\alpha\in I} A_\alpha,\ \exists\ i\in I,\ s.\ t.\ P\in A_i\in \tau,\ $
			\\
			则\ $\exists\  \varepsilon>0,\ s.\ t.\ P\in B_\varepsilon (P)\in A_i,\ $从而\ $B_\epsilon (P)\subset\bigcup\limits_{\alpha\in I} A_\alpha,\ $从而\ $\bigcup\limits_{\alpha\in I} A_\alpha\in\tau.\ $
		\end{itemize}
		所以,\ $\tau$\ 为\ $E^n$\ 的一个拓扑。
		\item[(2)] 
		$\forall\ P,Q\in E$\ 且\ $P\neq Q.\ $\ 则记\ $d=d(P,Q)$,\ 取\ $\varepsilon_1=\varepsilon_2=\frac{d}{3}$,\ 则\ $P\in B_{\varepsilon_1(P)}$(开),\ $Q\in B_{\varepsilon_2}(Q)$(开),且\ $B_{\varepsilon_1}(P)\cup B_{\varepsilon_2}(Q)=\emptyset$,\  
		\\
		否则,\ 若\ $\exists\ R\in B_{\varepsilon_1}(P)\cap B_{\varepsilon_2}(Q)$,\ 则\ $d(Q,R)<\frac{d}{3},d(Q,R)<\frac{d}{3},
		\\
		d=d(P,Q)\leq d(P,R)+d(Q,R)<\frac{2d}{3}<d$\ 矛盾!\ 所以\ $B_{\varepsilon_1}(P)\cup B_{\varepsilon_2}(Q)=\emptyset$\ 成立.
		\\
		从而,\ $E_n$\ 满足\ $T_2$\  分离性公理,\ 为\ Hausdorff\ 空间.
		\item[(3)] 
		取开集族\ $\mathscr{B}=\left\{ B_{\varepsilon}(P)|P\in \mathbb{Q}^n,\varepsilon\in \mathbb{Q} \right\} $,\ 其中\ $\mathbb{Q}$\ 为有理数,故\ $\mathscr{B}$\ 为可数的,下证其为拓扑基:
		\\
		$\forall \ P\in E_n,\quad
		\forall \ U=B_\varepsilon(P) \in N(P),\ 
		\\
		\exists\ \varepsilon'>0,$\ 且\ $\varepsilon'\in \mathbb{Q},\ s.\ t.\ \frac{3\varepsilon}{4}<\varepsilon'<\varepsilon.\quad 
		\exists\ Q \in \mathbb{Q}^n,\ s.\ t.\ \left|\overrightarrow{PQ}\right|<\frac{\varepsilon '}{4}$,
		\\
		令\ $B=B_{\varepsilon'}(Q),$\ 则\ $B\subset U\in $\ 且\ $B\in \mathscr{B}$.
		\\
		所以,\ $\mathscr{B}$\ 为\ $E^n$\ 中可数拓扑基,\ 从而\ $E^n$\ 第二可数。
	\end{itemize}
	
	
	
	\noindent
	\\
	{\textbf{1.4\ 证:}}
	\begin{itemize}
		\item[(1)]
		任取\ $E_n$\ 中直线\ $l$,\ 在\ $l$\ 上依次任取3个不同的点\ $P,Q,R$,\ 则有\ $\left|\overrightarrow{OQ}\right|=t\left|\overrightarrow{OP}\right|+\left|\overrightarrow{OR}\right|$,\ 其中\ $t\in (0,1)$.\ 记\ $\sigma(P)=P'\quad(\forall\ P\in E_n)$
		\\
		则\ $\left|\overrightarrow{O'Q'}\right|=\left|\overrightarrow{OQ}\right|=t\left|\overrightarrow{OP}\right|+(1-t)\left|\overrightarrow{OR}\right|=t\left|\overrightarrow{O'P'}\right|+(1-t)\left|\overrightarrow{O'R'}\right|\quad (t\in (0,1))$
		\\
		$\therefore\ P',Q',R'$\ 三点共线且保持分比,所以\ $\sigma$\ 将直线映为直线.
		\item[(2)]
		任取\ $E_n$\ 中两平行直线\ $l_1,l_2$,\ 则由(1)知\ $l_1,l_2$\ 在\ $\sigma$\ 下仍为直线,记为\ $l_1',l_2'$.\ 任取不同点\ $A,B\in l_1$,\ 不同点\ $C,D\in l_2$,\ 
		\\
		则\ $\overrightarrow{AB},\overrightarrow{CD}$\ 非零,且\ $\overrightarrow{AB}//\overrightarrow{CD}$\ 从而\ $\exists\ \lambda \neq 0,\ s.\ t.\ \overrightarrow{AB}=\lambda\ \overrightarrow{CD}$,
		\\
		而\ $\left|\overrightarrow{A'B'}\right|=\left|\overrightarrow{AB}\right|=\left| \lambda \right| \cdot \left|\overrightarrow{CD}\right|=\left| \lambda \right| \cdot \left|\overrightarrow{C'D'}\right|\quad\therefore \overrightarrow{A'B'}//\overrightarrow{C'D'}\quad\therefore\ l_1'//l'_2 $ 
		\\
		$\therefore\ $由\ $l_1,l_2$\ 任意性知\ $\sigma$\ 把\ $E_n$\ 中平行直线映为平行直线.
		\item[(3)] 
		记\ $\sigma (O)=O',\overrightarrow{OP_i}=\delta_i,\sigma(\delta_i)=\delta_i'=\overrightarrow{O'P'},i=1,2,...n$
		\\
		则由\ $\left\{O,\delta_i\right\}$\ 为正交标架知
		$$
		\left|\overrightarrow{OP_i}\right|\cdot\left|\overrightarrow{OP_j}\right|=
		\left\{ 
		\begin{array}{ll}
		{0,\ }&i\neq j,
		\\
		1, &i=j,
		\end{array}
		\right.
		\quad i,j=1,2,...,n
		$$
		$\therefore\ 
		\left\{
		O',\delta_i
		\right\}$
		\ 也为正交坐标系.
	\end{itemize}
	
	
	\noindent
	\\
	{\textbf{1.5\ 证:}}
	$\forall\ t\leq 0.\ 
	\forall\ x=(x_1,...,x_n),\ y=(y_1,...,y_n)\in \mathbb{R}^n$.
	\\
	由\ 
	$d(x,y)=\sqrt{\sum\limits_{i=1}^n (y^i-x^i)^2}$\ 
	知\ 
	$d(tx,ty)=td(x,y)$.\ 
	\\
	又由\ 
	$\sigma$\ 
	为等距变换知\ 
	$d(\sigma(tx),\sigma(ty)=d(tx,ty)=td(x,y)=td(\sigma(x),\sigma(y))$.\ 
	\\
	取\ 
	$y=\mathit{0}=(0,...,0)$,\ 
	则\ 
	$d(\sigma (tx),\sigma (\mathit{0})=t\cdot d(\sigma(x),\sigma (\mathit{0}))$,\ 
	\\
	则由\ 
	$\sigma $\ 
	保持共线性质而\ 
	$\overrightarrow{Ox}$\ 
	与\ 
	$\overrightarrow{O(tx)}$\ 
	共线,\ 知\ 
	$\overrightarrow{\sigma(\mathit{0})\sigma(x)}$\ 
	与\ 
	$\overrightarrow{\sigma (\mathit{0})}\sigma(tx)$\ 
	共线\ 
	\\
	$\therefore\ \overrightarrow{\sigma(\mathit{0})\sigma(tx)}=\pm t\overrightarrow{\sigma(\mathit{0})\sigma(x)}$\ 
	不妨取\ 
	$+t,(-t$\ 
	同理可证),\ 则有\ 
	\\
	$\sigma (tx)-\sigma(\mathit{0})=t(\sigma (x)-\theta(\mathit{0}))\Rightarrow \sigma (tx)=t\sigma(x)-(1-t)\sigma(\mathit{0})\cdots\cdots\cdots(1)$,\ 
	\\
	对(1)式左右两边关于\ t\ 求导得:\ 
	\begin{align*}
	\text{左边}&=\frac{\partial}{\partial t}\sigma_1(tx)
	\\
	&=(\frac{\partial}{\partial t}\sigma_1(tx),...\ ,\frac{\partial}{\partial t}\sigma_n(tx)) 
	\\
	&=(x_1,...,\ x_n)\cdot
	\left(
	\begin{array}{ccc}
	\partial_1\sigma_1(tx)  &  \cdots & \partial_1\sigma_n(tx)
	\\
	\colon                  &\        &\colon
	\\
	\partial_n\sigma_1(tx)  &\cdots   &\partial_n\sigma_n(tx)
	\end{array}
	\right)
	\\
	\text{左边}&=\sigma(x)-\sigma (\mathit{0})
	\end{align*}
	则左边=右边,且令\ t=0\ 后有:\
	$$
	\sigma(x)=\sigma (\mathit{0})+
	(x_1,...,\ x_n)\cdot
	\left.
	\left(
	\begin{array}{ccc}
	\partial_1\sigma_1(tx)  &  \cdots & \partial_1\sigma_n(tx)
	\\
	\colon                  &\        &\colon
	\\
	\partial_n\sigma_1(tx)  &\cdots   &\partial_n\sigma_n(tx)
	\end{array}
	\right)
	\right|_{t=0}                                              
	$$
记\ 
$a_0^j=\sigma_i(0)\qquad a_i^j=(\partial_i\sigma_j)|_(t=0)\qquad$
其中\ 
$ i=1,2,...,n.\quad j=1,2,...,n.\  \text{有:}\ $
$$
\sigma(x_1,...x_n)=(a_0^1,...,a_0^n)+(x_1,...,x_n)\cdot
                    \left(
                    \begin{array}{ccc}
                    a^1_1 &\cdots &a_1^n
                    \\
                    \colon&\      &\colon
                    \\
                    a_n^1 &\cdots &a_n^n
                    \end{array}
                    \right)
$$
取\
$\varepsilon_i=(0,...,\overset{\overset{\text{第i个}}{\downarrow}}{1},0,...0)$,\
则,\
\\
$1=d(\varepsilon_i,0)=d(\sigma(\varepsilon_i),\sigma(\mathit{0}))\Rightarrow (a_i^1)^2+(a_i^2)^2+\cdots+(a_i^n)^2=1\ (\forall\ i=1,2,...,n)$
\\
$(i\neq j\ \text{时})$
\begin{align*}
2=1^2+1^2&=d^2(\varepsilon_i,\varepsilon_j)
\\
         &=d^2(\sigma(\varepsilon_i),\sigma(\varepsilon_j))
\\
         &=(a_j^1-a_i^1)^2+\cdots +(a_j^n-a_i^n)^2
\\
         &=((a_i^1)^2+\cdots+(a_i^n)^2)+((a_j^1)^2+\cdots+(a_j^n)^2)-2(a_j^1a_i^1+\cdots+a_j^n a_i^n)
\\
         &=1+1-2(a_j^1a_i^1+\cdots+a_j^n a_i^n)
\\
\Rightarrow
a_j^1a_i^1+\cdots+a_j^n a_i^n=0\ (\ i\neq j\ \text{时}\ )
\end{align*}
$\text{从而}\ (a_i^j)_{n\times n}\text{为单位正交矩阵. } $
\\
\\
\\	
\noindent
{\textbf{1.6\ 证:}}
设\ Q\ 关于\
$\left\{
O;\delta_i
\right\}$\
的坐标为:\
$x=(x^1,...\ ,x^n)$,\
即\
$Q-O=x^1\delta_1+\cdots+x^n\delta_n$.\
\\
由第5题知\
$\sigma$\
为线性变换,而\
$\sigma(O)=P,\ \sigma(\delta_i)=e_i$,\
\\
$\therefore\ \sigma(Q)-\sigma(O)=\sigma(Q-O)=\sigma(x^1\delta_1+\cdots+x^n\delta_n)=x^1\sigma(\delta_1)+\cdots+x^n\sigma(\delta_n)=x^1 e_1+\cdots+x^n e_n$\
即点\ $Q'$\ 关于\
$\left\{
P;e_i
\right\}$\
的坐标等于点\ $Q$\ 关于\
$\left\{
O;\delta_i
\right\}$\ 的坐标.
\\
\\
\\
\noindent
{\textbf{1.7\ 证:}}设\ $\left\lbrace O,\delta_i \right\rbrace $\ 为\ $E_n$\ 中某一直角坐标系,\ $f$\ 在\ $\left\lbrace O,\delta_i \right\rbrace $\ 之下表示成
\begin{align*}
\overrightarrow{Of(t)}&=\sum_{i=1}^{n}x^i(t)\delta_i,\ \forall\ t\in \mathbb{R}
\\
                      &=(\delta_1,...,\delta_n)  \left( \begin{array}{c}
                                                             x^1(t)
                                                              \\
                                                              \colon
                                                              \\
                                                              x^n(t)                                          
                                                        \end{array}  \right)
\end{align*}
任取另一直角坐标系\ $\left\lbrace P,e_i \right\rbrace $\ 则有唯一表示:
$$\overrightarrow{OP}=\sum_{i=1}^{n}a^i\delta_i=(\delta_1,...,\delta_n)
                                                      \left(        \begin{array}{c}
                                                                    a^1(t)
                                                                    \\
                                                                    \colon
                                                                    \\
                                                                    a^n(t)   
                                                                    \end{array}  \right)
 \qquad
 e_i=\sum_{i=1}^{n}a^i\delta_i=(\delta_1,...,\delta_n)  \left( \begin{array}{c}
                                                                  a^1_i
                                                                  \\
                                                                  \colon
                                                                  \\
                                                                  a^n_i
                                                               \end{array}  \right)
                                                               \quad (i=1,2,...n)                                           
$$
从而
$$
(e_1,...e_n)=(\delta_1,...\delta_n) \left( \begin{array}{ccc}
                                           a^1_1  &  \cdots  &  a^1_n
                                           \\ 
                                           \colon &          &  \colon
                                           \\
                                           a^n_1  &  \cdots  &  a^n_n
                                           \end{array} \right)
$$
记\ $(a_j^i)=A.$\ 由\ $(e_1,...e_n)$\ 与\ $(\delta_1,...,\delta_n)$\ 均为正交向量组知\  $|A|\neq 0.
\\
\therefore A$\ 可逆。记\ $A^{-1}=(b_j^i)_{n\times n}\ \therefore (\delta_1,...,\delta_n)=(e_1,...e_n)A^{-1}$
$$\overrightarrow{Of(t)}=(\delta_1,...,\delta_n)
                                           \left(        \begin{array}{c}
                                          x^1(t)
                                          \\
                                          \colon
                                          \\
                                          x^n(t)   
                                          \end{array}  \right)
                               =(e_1,...e_n)A^{-1} 
                                          \left(        \begin{array}{c}
                                          x^1(t)
                                          \\
                                          \colon
                                          \\
                                          x^n(t)   
                                          \end{array}  \right)
      $$
                    \\
                    $$
                      \overrightarrow{OP}=(\delta_1,...,\delta_n)
                                                              \left(        
                                                              \begin{array}{c}
                                                                 a^1(t)
                                                              \\
                                                              \colon
                                                              \\
                                                              a^n(t)   
                                                              \end{array}  
                                                              \right)
                                        =(e_1,...e_n)A^{-1}
                                                          \left(        
                                                         \begin{array}{c}
                                                         a^1(t)
                                                         \\
                                                         \colon
                                                         \\
                                                         a^n(t)   
                                                         \end{array}  
                                                         \right)   
$$
而\ $\overrightarrow{Of(t)}=\overrightarrow{OP}+\overrightarrow{Pf(t)}$\ 从而
\begin{align*}
\overrightarrow{Pf(t)}&=\overrightarrow{Of(t)}-\overrightarrow{OP}
                      \\
                      &= (e_1,...e_n)A^{-1} 
                                           \left( 
                                            \begin{array}{c}
                                           x^1(t)
                                           \\
                                           \colon
                                           \\
                                           x^n(t)   
                                           \end{array}  
                                           \right)       
                                                                  -
                                                                       (e_1,...e_n)A^{-1}
                                                                                         \left( 
                                                                                         \begin{array}{c}
                                                                                         a^1(t)
                                                                                         \\
                                                                                         \colon
                                                                                         \\
                                                                                         a^n(t)   
                                                                                         \end{array}  
                                                                                         \right)              
                      \\
                      &=(e_1,...e_n)A^{-1}
                                           \left(  
                                           \begin{array}{c}
                                           x^1(t)- a^1(t)
                                           \\
                                           \colon
                                           \\
                                           x^n(t)-a^n(t)   
                                           \end{array}  
                                           \right)  
                      \\
                      &\overset{\bigtriangleup}{=}\sum_{k=1}^{n}e_ky^k(t) \qquad\text{其中}\ y^k(t)=\sum_{j=1}^{n}b_j^k( x^j(t)-a^j(t) )  \quad (k=1,2,...,n)  
\end{align*}
由\ $x^j(t)$\ 连续(或r次连续可微)可得\ $y^k(t)$\ 连续(或r次连续可微),从而映射\ $f:\mathbb{R}\rightarrow E^n$\ 的连续性和r次连续可微性与\ $E^n$\ 中直角坐标系的选取无关。 
	


\noindent
\\
{\textbf{1.8\ 证:}}	
	记\ $f(t)$\ 在\ $\left\lbrace O;\delta_i \right\rbrace $\ 与\ $\left\lbrace P;e_i \right\rbrace $\ 下的坐标为\ $(x^1(t),...,x^n(t)),(y^1(t),...,y^n(t)),$\ 其中
	\\
	$$\overrightarrow{OP}=\sum_{i=1}^{n}a_i\delta_i,\qquad e_i=\sum_{j=1}^{n}a_i^j\delta_j\quad(i=1,...n)$$
	\\
	由的(1.14)
	$$x^i(t)=a^i+\sum_{j=1}^{n}y^j(t)a^i_j$$
	则等式两边同时对t求导有
	$$\frac{dx^i}{dt}(t_0)=\sum_{j=1}^{n}\frac{dy^j(t)}{dt}a_j^i$$
	又由(2.6)
	\begin{align*}
	f'(t_0)&=\sum_{i=1}^{n}\frac{dx^i}{dt}(t_0)\delta_i
           \\
	       &=\sum_{i=1}^{n}\sum_{j=1}^{n}\frac{dy^j}{dt}a_j^i\delta_i
	       \\
	       &=\sum_{j=1}^{n}\left(\frac{y^j(t)}{dt}\sum_{i=1}^{n}a^i_j\delta_i\right)
	       \\
	       &=\sum_{j=1}^{n}\frac{dy^j}{dt}e_j
	\end{align*}
	从而可见其形式不变,即切向量定义式(2.6)与直角坐标系的选取无关。
\noindent
\\
\\
{\textbf{1.9\ 证:}}
	\begin{itemize}
		\item [(1)]
		\begin{align*}
		D_v(g+\lambda h)&=\left\langle \triangledown (g+\lambda h)(P),v\right\rangle 
		              \\
		                &=\left\langle \triangledown g(P)+\lambda \triangledown h(P),V\right\rangle 
		                \\
		                &=\left\langle \triangledown g(P),v \right\rangle +\lambda \left\langle \triangledown h(P),v \right\rangle
		                \\
		                &=D_v(g)+\lambda D_v(h)
		\end{align*}
		\item [(2)]
		\begin{align*}
		D_v(g\cdot h)&=\left\langle \triangledown (gh)(P),v \right\rangle 
		             \\
		             &=\left\langle ((\triangledown g)h)(P),v \right\rangle +\left\langle (g(\triangledown h))(P),v \right\rangle
		             \\
		             &=h(P)\left\langle \triangledown g(P),v \right\rangle +g(P)\left\langle \triangledown h(P),v \right\rangle
		             \\
		             &=h(P)D_v g+g(P)D_vh
		\end{align*}
	\end{itemize}



\noindent
\\
\\
{\textbf{1.10\ 证:}}
	$E_n\rightarrow R$\ 上的函数\ $x^i:P=\lambda^1\delta_1+...+\lambda^n\delta_n\rightarrow \lambda^i\quad(i=1,2,...,n)$
	\\
	$\forall\ P=(\lambda^1,...,\lambda^n)\in E^n,$\ 在\ $P$\ 的邻域内取\ $Q=P+\triangle P=(\lambda^1+\triangle\lambda^1,...,\lambda^n+\triangle\lambda^n)$,\ 则
	$$
	\lim\limits_{\triangle\lambda^j\to 0}\frac{x^i(Q)-x^i(P)}{\triangle \lambda^j}=\lim\limits_{\triangle\lambda^j\to 0}\frac{\triangle \lambda^i}{\triangle \lambda^j}=\delta_i^j\in C^{\infty}
	$$ 
	从而\ $x^i\in C^\infty\quad (\forall\  i=1,2,...,n).$
	
\noindent
\\
\\
{\textbf{1.11\ 证:}} 
\begin{itemize}
\item [(1)]
在\ $E^m$\ 中,取两新旧直角坐标分别为\ $\{O;\delta_i\},\{P;e_i\},$\ 且满足\ 
$$
\left\{
\begin{array}{c}
       \overrightarrow{OP}=\sum_{i=1}^{m}a^i\delta_i
       \\
       e_j=\sum_{i=1}^{m}a_j^i\delta_i
\end{array}
\qquad j=1,..,m.\right.
$$
 在\ $E^n$\ 中,取两新旧直角坐标分别为\ $\{O;\xi_i\},\{P;\eta_i\},$\ 且满足\ 
$$
\left\{
\begin{array}{c}
\overrightarrow{OQ}=\sum_{i=1}^{m}b^i\xi_i
\\
\eta_j=\sum_{i=1}^{m}b_j^i\xi_i
\end{array}
\qquad j=1,..,m.\right.
$$
记原映射为\ $F(\lambda^1,..,\lambda^m)=(f^1(\lambda^1,..,\lambda^m),...,f^n(\lambda^1,..,\lambda^m)).$
\\
设在新坐标下表示为\ $G(\mu^1,...,\mu^m)=(g^1(\mu^1,..,\mu^m),...,g^n(\mu^1,..,\mu^m))$.
\\
则有\ $f^l=b^l+\sum_{j=1}^{n}b^l_jg^j\quad \lambda^k=a^k+\sum_{j=1}^{m}a^k_j\mu^j\ (k=1,..,m).$\ 再记\ $(a^k_j)_{m\times m}$\ 的逆为\ $(C_k^j)_{m\times m}$
\begin{align*}
\therefore \frac{\partial f^l}{\partial \lambda^k}
    =&\sum_{j=1}^{n}b^l_j\frac{\partial g^j}{\partial\lambda^k}
    \\
    =&\sum_{j=1}^{n}b_j^l(\sum_{i=1}^m\frac{\partial g^j}{\partial \mu^i}C_k^i)
    \\
    =&\sum_{j=1}^{n}sum_{i=1}^mb_j^lfrac{\partial g^j}{\partial \mu^i}C_k^i
\end{align*}
$$\therefore (\frac{\partial f^l}{\partial \lambda^k})_{m\times n}=A^{-1}(\frac{\partial g^j}{\partial \mu^i})_{m\times n}B$$
\\
记\ $J_f,J_g$\ 分别为\ $F(\lambda^1,..,\lambda^m)$\ 与\ $G(\mu^1,..,\mu^m)$\ 的Jacobi矩阵。则有\ $J_f=A^{-1}J_gB$.
\item [(2)]
 由于A、B可逆,所以\ $A^{-1},B$\ 均可表为若干初等行、列变换的乘积,
 \\
  $\therefore r(J_f)=r(J_g)$,\ 在任意点\ $x_0.$
\item [(3)]
	在\ $E^m$\ 中任取点P,记\ $Q=f(P)$.
	\\
	在\ $P$\ 点邻域分别取两新旧曲纹坐标系\ $(u^1,...,u^m),(v^1,...,v^m)$.\
	两者之间由同胚映射\ $g$\ 关联:\ $v^i=g^i(u^1,...,u^m)\quad i=1,..,m. \qquad $ 记\ $g^{-1}=\bar{g}$\ 则\ $u^i=\bar{g}^i(v^1,..,v^m)\quad i=1,..,m.$
	\\
	同样在\ $Q$\ 点邻域分别取两新旧曲纹坐标系\ $(s^1,...,s^n),(t^1,...,t^n)$.\
	两者之间由同胚映射\ $h$\ 关联:\ $t^i=h^i(s^1,...,s^n)\quad i=1,..,m$
	\\
	原函数\ $f(u^1,..,u^m)$\ 中分量记为\ $f^i(u^1,..,u^m)\ (i=1,..,n)$\ 在\ $E^m$\ 与\ $E^n$\ 间曲纹坐标变换下为\ $\tilde{f}(v^1,..,v^m)$,\ 其在新坐标下分量为:	$\tilde{f}^i(v^1,..,v^m)=h^i(f^1,..,f^n)\quad $其中\ $f^j(u^1,..,u^m)=f^j(\bar{g}(v^1,..,v^m),...,\bar{g}(v^1,..,v^m))\ (i=1,..,n;j=1,..,n)$
	\\
	从而在变换后Jacobi矩阵\ $J_{n\times m}$\ 为
	\begin{align*}
	J_j^i&=\frac{\partial\tilde{f}^i}{\partial v^j}
	\\
	     &=\frac{\partial h^i}{\partial f^k}\frac{\partial f^k}{\partial u^r}\frac{\partial \bar{g}^r}{\partial v^j}\quad(i,k=1,..,n,r,j=1,..,m)
	\end{align*}
	在\ P\ 点,记\ $g$\ 的雅克比矩阵\ $(\frac{\partial g^i}{\partial u^j})_{m\times m}$\ 为\ G,\ $h$\ 的雅克比矩阵\ $(\frac{\partial h^i}{\partial v^j})_{n\times n}$\ 为\ H,\ 变换前\ $f$\ 的雅克比矩阵\ $(\frac{\partial f^i}{\partial u^j})_{n\times m}$\ 为\ $J_0$\ 则有:
	$$J=HJ_0G^{-1}$$
\end{itemize}

\noindent
\\
\\
{\textbf{1.12\ 证:}} 记原方程组为$x^i=f^i(u^1,..,u^n),\ i=1,..,n.$\ 则\ $f=f^i\delta_i$\ 的雅克比矩阵的行列式为
\begin{align*}
     \left| \frac{\partial(f^1,..,f^n)}{\partial(u^1,..,u^n)} \right|
      &=\left|
          \begin{array}{ccccccc}
          -x^1tanu^1      &-x^1tanu^2      &\cdots   &-x^1tanu^{n-2}  &-x^1tanu^{n-1} &\frac{x^1}{u^n}
          \\
          -x^2tanu^1      &-x^2tanu^2      &\cdots   &-x^2tanu^{n-2}  &x^2cotu^{n-1}  &\frac{x^2}{u^n}
          \\
          \colon          &\colon      	   &\colon   &\colon          &\colon         &\colon
          \\
          -x^{n-1}tanu^1  &x^{n-1}cotu^2   &\cdots   &  0             &0              &\frac{x^{n-1}}{u^n}
          \\
          x^{n}cotu^1     &0               &\cdots   &  0             &0              &\frac{x^{n}}{u^n}
          \end{array}
     \right|
     \\
     &=x^1x^2\cdots x^n
       \left|
        \begin{array}{cccccccc}
        -tanu^1 &-tanu^2 &\cdots &-tanu^{n-2} &-tanu^{n-1} &1
        \\
        -tanu^1 &-tanu^2 &\cdots &-tanu^{n-2} &cotu^{n-1}  &1
        \\
        -tanu^1 &-tanu^2 &\cdots &cotu^{n-2}  &0           &1
        \\
        \colon  &\colon  &\colon &\colon      &\colon      &\colon
        \\
        -tanu^1 &\cot u^2&\cdots &0           &0           &1
        \\
        \cot u^1&0       &\cdots &0           &0           &1
        \end{array}
       \right|
    \\
    &=\frac{x^1 x^2 \cdots x^n}{u^n} \cdot \left[ (-1)^{1+n}  (-1)^{\frac{(n-1)(n-2)}2{}} cotu^1 \cdots cotu^{n-1}\right.
    \\&\left.\qquad\qquad\qquad\qquad+(-1)^{2+n} (-1)^{\frac{(n-1)(n-2)}{2}} cotu^1 \cdots cotu^{n-2}(-tanu^{n-1})\right]
    \\
    &=(-1)^{\frac{n^2-n+4}{2}} \left(\prod_{i=1}^{n-2} r^i \prod_{j=1}^{i}cosu^j\right) x^1x^2(u^n)^{n-3}\frac{1}{sinu^{n-1}cosu^{n-1}}
\end{align*}
当\ $x^1,x^2\neq 0$\ 时,有\ $u^n,cosu^1,\cdots,cosu^{n-1},sinu^{n-1}$\ 非零,又由\ $r^1,\cdots r^n$\ 为正数可知:$\left| \frac{\partial(f^1,..,f^n)}{\partial(u^1,..,u^n)} \right|\neq 0$,\ 即\ f\ 的秩为\ n\ .\ 从而\ $(u^1,\cdots u^n)$\ 给出\ $E^n$\ 中除坐标面\ $\left\lbrace  (0,0,x^3,\cdots,x^n):x^3,\cdots,x^n\in\ \mathbb{R} \right\rbrace $\ 以外的任意一点的邻域内的曲纹坐标系。



\noindent
\\
\\
{\textbf{1.13\ 解:}} 设\ $(x,y,z)$ 是\ $E^3$ \ 中的直角坐标系,令
$$
\left\lbrace 
           \begin{array}{l}
           x=\rho \cos\psi\cos \theta
           \\
           y=\rho \cos\psi\sin \theta
           \\
           z=\rho \sin\psi
           \end{array}
\right. \quad(0<\rho <+\infty,-\frac{\pi}{2}<\psi<\frac{\pi}{2},-\pi<\theta<\pi )
$$
则\ $(\rho,\psi,\theta)$\ 给出球坐标系,记\ $u^1=\rho ,u^2=\psi ,u^3=\theta $\ 则可求其自然标架场如下:
\begin{align*}
\overrightarrow{r_1}&=(cos\psi cos\theta ,cos \psi sin \theta ,sin\psi )
\\
\overrightarrow{r_2}&=(-\rho sin \psi cos\theta ,-\rho sin\psi sin\theta ,\rho cos\psi )
\\
\overrightarrow{r_3}&=(-\rho cos\psi sin\theta ,\rho cos\psi cos\theta ,0)
\end{align*}
则\ $\forall \ Q=r(P),\{Q,r_i\}$\ 为\ $E^3$\ 中球坐标系诱导的自然标架场。
\\
下求其度量系数:
\begin{align*}
&g_{11}=\langle \overrightarrow{r_1},\overrightarrow{r_1} \rangle=1  & g_{12}=g_{21}=\langle \overrightarrow{r_1},\overrightarrow{r_2} \rangle =0     
\\
&g_{31}=g_{13}=\langle \overrightarrow{r_1},\overrightarrow{r_3} \rangle=0 
\\
&g_{22}=\langle \overrightarrow{r_2},\overrightarrow{r_2} \rangle=\rho ^2  & g_{23}=g_{32}=\langle \overrightarrow{r_2},\overrightarrow{r_3} \rangle =0     
\\
&g_{33}=\langle \overrightarrow{r_3},\overrightarrow{r_3} \rangle=\rho ^2cos^2\psi  
\end{align*}
$\therefore$\ 度量系数为\ $g_{11}=\rho^2,\ g_{33}=\rho^2cos^2\psi, \ g_{ij}=0\ (i\neq j\text{\ 时,其中}\ i,j=1,2,3).$
\\
则\ $g^{11}=1,\ g^{22}=\frac{1}{\rho^2},\ g^{33}=\frac{1}{\rho^2cos^2\psi },\  g^{ij}=0\ (i\neq j\text{\ 时,其中}\ i,j=1,2,3)$.
\\
且\ $\frac{\partial g_{11}}{\partial u^i}=0\ (i=1,2,3)\quad \frac{\partial g_ij}{\partial u^k}\ (i,j,k=1,2.3)\quad \frac{\partial g_{22}}{\partial u^1}=2\rho \quad \frac{\partial g_{22}}{\partial u^2}=0 \quad\frac{\partial g_{22}}{\partial u^3}=0 \quad\frac{\partial g_{33}}{\partial u^1}=2\rho cos^2\psi  \quad \frac{\partial g_{33}}{\partial u^2}=-\rho^2 sin2\psi  \quad\frac{\partial g_{33}}{\partial u^3}=0$
\\
从而由\ 
$$
\varGamma^k_{il}=\frac{1}{2}g^{kj}(\frac{\partial g_{ij}}{\partial u^l}+\frac{\partial g_{jl}}{\partial u^i}-\frac{\partial g_{il}}{\partial u^j})
$$
得:
\begin{align*}
\varGamma^1_{33}&=\frac{1}{2}g^{11}(\frac{\partial g_{31}}{\partial u^3}+\frac{\partial g_{13}}{\partial u^1}-\frac{\partial g_{33}}{\partial u^1})=\rho cos^2\psi
\\
\varGamma^2_{33}&=\frac{1}{2}g^{22}(\frac{\partial g_{32}}{\partial u^3}+\frac{\partial g_{32}}{\partial u^1}-\frac{\partial g_{33}}{\partial u^2})=-sins\psi 
\\
\varGamma^3_{33}&=\frac{1}{2}g^{33}(\frac{\partial g_{33}}{\partial u^3}+\frac{\partial g_{33}}{\partial u^3}-\frac{\partial g_{33}}{\partial u^3})=0
\end{align*}
又由
\begin{align*}
 \frac{\partial \overrightarrow{r_1}}{\partial u^1}=0\qquad \qquad \qquad  &\frac{\partial \overrightarrow{r_1}}{\partial u^2}=\frac{1}{\rho}\overrightarrow{r_2}  &\frac{\partial \overrightarrow{r_1}}{\partial u^3}=\frac{1}{\rho}\overrightarrow{r_3}
\\
     &\frac{\partial \overrightarrow{r_2}}{\partial u^2}=-\rho \overrightarrow{r_1} &\frac{\partial \overrightarrow{r_2}}{\partial u^3}=-tan\psi  \overrightarrow{r_3}
\end{align*}
从而\ Christoffel\ 系数为:
\begin{align*}
&\varGamma^2_{12}=\varGamma^2_{21}=\frac{1}{rho} &\varGamma^3_{31}=\varGamma^3_{13}=\frac{1}{rho}
\\
&\varGamma^1_{22}=-rho &\varGamma^3_{23}=\varGamma^3_{32}=-tan\psi 
\\
&\varGamma^1_{33}=-\rho cos^2\psi  &\varGamma^2_{33}=-sin^2\psi 
\end{align*}
其余为\ 0.

\noindent
\\
\\
{\textbf{1.14\ 解:}}
设\ (x,y,z)\ 为\ $E^3$\ 中直角坐标系,设
\begin{displaymath}
\left\lbrace  
\begin{array}{l}
	x=\rho cos \theta
	\\
	y=\rho sin \theta
	\\
	z=t
\end{array}
\right.
     \quad(\text{其中}\ 0<r<+\infty ,\ -\pi <\theta <\pi ,\ -\infty <t<+\infty )
\end{displaymath}
则\ $(\rho ,\theta ,t)$\ 为柱坐标系。记\ $u^1=\rho ,\ u^2=\theta,\ u^3=t$,\ 则有:
\begin{align*}
&\overrightarrow{r_1}=(cos\theta ,sin\theta ,0)
\\
&\overrightarrow{r_2}=(-\rho sin\theta ,\rho cos\theta ,0)
\\
&\overrightarrow{r_3}=(0,0,1)
\end{align*}
且\ $g_{11}=1,\ g_{22}=\rho^2,\ g_{33}=1,\ g_{ij}=0\,(i\neq j\text{时})$
由于:
\begin{displaymath}
\begin{array}{ccc}
	\frac{\partial \overrightarrow{r_1}}{\partial u^1}=0  & 
	\frac{\partial \overrightarrow{r_1}}{\partial u^2}=\frac{1}{\rho}\overrightarrow{r_2}  &
	\frac{\partial \overrightarrow{r_1}}{\partial u^3}=0
	\\
	 &
	\frac{\partial \overrightarrow{r_2}}{\partial u^2}=-\rho \overrightarrow{r_1} &
	\frac{\partial \overrightarrow{r_2}}{\partial u^3}=0
	\\
	 &
	 &
	\frac{\partial \overrightarrow{r_3}}{\partial u^3}=0
\end{array}
\end{displaymath}
所以\ Christoffel\ 系数为:
$$\varGamma^2_{12}=\varGamma^2_{21}=\frac{1}{\rho}\quad\varGamma^1_{22}=-\rho $$
其余为0


\noindent
\\
\\
{\textbf{1.15\ 证:}}
$\because\ v=v^i\overrightarrow{e_i}$
\begin{align*}
\therefore\ dv&=v^jd\overrightarrow{e_j}+\overrightarrow{e_i}dv^i
              \\
              &=v^j(\frac{\partial \overrightarrow{e_j}}{\partial u^k}du^k)+\overrightarrow{e_i}dv^i
              \\
              &=v^j\varGamma^i_{jk}\overrightarrow{e_i}du^k+\overrightarrow{e_i}dv^i
              \\
              &=(v^j\varGamma^i_{jk}du^k+dv^i)\overrightarrow{e_i}
\end{align*}
$\therefore\ v\text{为平行向量场}\Leftrightarrow v^j\varGamma^i_{jk}+dv^i=0(\forall\ i=1,2,...,n)$

\noindent
\\
\\
{\textbf{1.16\ 证:}}
定义加法\ $'+':(f+g)(x)=f(x)+g(x)$,\ 数乘:\ $(\lambda f)(x)=\lambda f(x) $
\\
先证\ $(\mathscr{L}(v_1,\cdots,v_r;\mathbb{R}),+)$\ 构成加法群。
\begin{itemize}
	\item [(1)] $\forall\ f,g\in \mathscr{L}(v_1,\cdots ,v_r;\mathbb{R})\quad \forall\ i\in \{1,\cdots ,r\}\quad\forall \ x_1=(x^1,\cdots,x_1^i,\cdots,  x^r),x_2=(x^1,\cdots ,x_2^i,\cdots, x^r)\in V_1\times \cdots \times V_r\quad \forall \ k_1,k_2\in \mathbb{R}$\ 有
	\begin{align*}
	 &(f+g)(x^1,\cdots ,k_1x^i_1+k_2x^i_2,\cdots x^r)
	 \\
	=&f(x^1,\cdots ,k_1x^i_1+k_2x^i_2,\cdots x^r)+g(x^1,\cdots ,k_1x^i_1+k_2x^i_2,\cdots x^r)
	\\
	=&(k_1f(x_1)+k_2f(x_2))+(k_1g(x_1)+k_2g(x_2))
	\\
	=&k_1(f+g)(x_1)+k_2(f+g)(x_2)
	\end{align*}
	$\therefore f+g\in \mathscr{L}(V_1,\cdots,V_r;\mathbb{R})$
	\item [(2)] $\forall\ f,g,h\in \mathscr{L}(V_1,\cdots,V_r;\mathbb{R})\quad \forall\ x\in V_1\times ,\cdots ,V_r$,\ 有
	$$[(f+g)+h](x)=(f+g)(x)+h(x)=f(x)+g(x)+h(x)$$
	$$[f+(g+h)](x)=f(x)+(g+h)(x)=f(x)+g(x)+h(x)$$
	$\therefore (f+g)+h=f+(g+h)$
	\item[(3)] 取\ $\theta $\ 为\ $\forall\ x\in V_1\times\cdots \times V_r,\theta (x)=0$
	\\
	则显然\ $f+\theta =\theta +f,\forall\ f$
	\item [(4)] $\forall\ f,$\ 取\ $-f$\ 为\ $(-f)(x)=-f(x)$\ 则\ $f+(-f)=(-f)+f=\theta$
	\item [(5)] $\forall\ f,g\in\mathscr{L}(V_1\cdots V_r;\mathbb{R})$\ 显然\ $f+g=g+f.$
\end{itemize}
所以\ $(\mathscr{L}(v_1,\cdots,v_r;\mathbb{R}),+)$\ 构成Abel群。
$\forall\ \lambda,\mu\in \mathbb{R}\quad \forall f,g\in \mathscr{L}(V_1\times \cdots \times V_r;\mathbb{R})$,\ 又有
\begin{itemize}
	\item [(6)] $(\lambda,\mu)f=\lambda f+\mu f$
	\item [(7)] $(\lambda \mu )g=\lambda (\mu g)$
	\item [(8)] $\lambda (f+g)=\lambda f+\lambda g$
	\item [(9)] $1\cdot f=f$
\end{itemize}
所以\ $\mathscr{L}(V_1,\cdots ,V_r;\mathbb{R})$\ 构成\ $\mathbb{R}$\ 上的线性空间。


\noindent
\\
\\
{\textbf{1.17\ 证:}}
\begin{itemize}
	\item [(1)]
	$\forall\ k_1,k_2\in \mathbb{R}\ u_1,u_2\in V,$\ 则由\ $\tilde{f}$\ 为2重线性函数知
	\begin{align*}
	f(k_1u_1+k_2u_2)=&\sum_{i=1}^{n}\tilde{f}(k_1u_1+k_2u_2,\delta^i)\delta_i
	                \\
	                =&\sum_{i=1}^{n}(k_1\tilde{f}(u_1,\delta^i)+k_2\tilde{f}(u_2,\delta^i))\delta_i
	                \\
	                =&k_1\sum_{i=1}^{n}\tilde{f}(u_1,\delta^i)\delta_i+k_2\sum_{i=1}^{n}\tilde{f}(u_2,\delta^i)\delta_i
	                \\
	                =&k_1f(u_1)+k_2f(u_2)
	\end{align*}
	所以\ $f$\ 为线性变换。
	\item [(2)]
	若另取基底\ $\{e^i\}$,\ 且\ $e^i=a^i_j\delta_j$\, 其对偶基底\ $\{e_i\}$,\ 则有\ $\delta_j=a^i_je_i$ ,\ 其中\ $(b^j_i)$\ 为\ $(a_i^j)$\ 的逆矩阵。且由
	\begin{align*}
	\delta^i(e_j)=\delta_i(a^k_j\delta_k)=a^k_j\delta_k^i=a^i_j=a_k^ie^k(e_j)
	\end{align*} 
	知\ $\delta^i=a^i_ke^k.$
	\\
	$\therefore f(u)=\tilde{f}(u,\delta^i)\delta_i=\tilde{f}(u,a^i_ke^k)(b^j_ie_j)=a^i_kb^j_i\tilde{f}(u,e^k)e_j=\delta_k^j\tilde{f}(u,e^k)e_j=\tilde{f}(u,e^j)e_j$
	\\
	$\therefore\ f$\ 的定义与基底的选取无关。
\end{itemize}


\noindent
\\
{\textbf{1.18\ 证:}}
取定\ $t,\forall \ v=(v_1,\cdots v_r) \in V\times \cdots \times V\quad v^1\in V^\star$,\ 取\ $F:V^\star \times V_1\cdots \times V_r \to \mathbb{R}$\ 为\ $F(v^1,v_1,\cdots\ v_r)=v^1(t(v))$\ ,显然\ $F$\ 是\ $V^\star\times V \times \cdots V$\ 上的一个\ $1+r$\ 重线性函数,即\ $(1,r)$\ 型张量。 
\\
反之,若给定一个\ $(1,r)$\ 型张量\ $F:V^\star\times V_1\times \cdots V_r,\ \forall \ v=(v1,\cdots, v_r)\in V\times \cdots \times \to \mathbb{R}  $,\ 令\ $t(v)=F(\delta^i,v_1,\cdots ,v_r)\delta_i$,\ 显然\ $t$\ 为\ $r$\ 重线性映射。
\\
所以\ $t$\ 等同一个\ $(1,r)$\ 型张量。




\noindent
\\
\\
{\textbf{1.19\ 证:}}
\begin{itemize}
	\item [(1)]
	$\forall \ \alpha_1,\cdots ,\alpha_2\in \mathscr{L}(V_1,\cdots ,V_p;\mathbb{R}),\quad \forall \ \beta \in \mathscr{L}(W_1,\cdots ,W_q;\mathbb{R}) $
	\\
	$  \forall \ v\in V_1\times \cdots \times V_p,\quad w\in W_1\times \cdots \times W_q$ 
	\begin{align*}
	  &(\alpha_1+\alpha_2)\otimes\beta (v,w)
	\\
	=&(\alpha_1+\alpha_2)(v)\cdot \beta (w)
	\\
	=&[\alpha(_1(v)+\alpha_2(v)]\cdot \beta(w)
	\\
	=&\alpha_1(v)\cdot \beta(w)+\alpha_2(v)\cdot \beta(w)
	\\
	=&\alpha_1\otimes \beta (v,w)+\alpha_2 \otimes \beta (v,w)
	\\
	=&(\alpha_1\otimes\beta +\alpha_2\otimes\beta )(v,w)
	\end{align*}
	$\therefore \ (\alpha_1+\alpha_2)\otimes \beta =\alpha_1\otimes \beta +\alpha_2\otimes \beta $
	同理\ $\beta\otimes(\alpha_1+\alpha_2)=\beta \otimes \alpha_1+\beta \otimes \alpha_2$
	\item [(2)]
	$\forall \ \alpha \in \mathscr{L}(V_1,...,Vr;\mathbb{R}),\ \beta \in \mathscr{L}(W_1,...,W_s;\mathbb{R}),\ \gamma \in \mathscr{L}(Z_1,...,Z_t;\mathbb{R})$
	\\
	$\forall \ V_1\times \cdots \times V_r,\ w\in W_1\times \cdots \times W_s,\ z\in Z_1\times \cdots \times Z_t$,\ 则
	\begin{align*}
	 &(\alpha \otimes \beta )\otimes\gamma (v,w,z)
	\\
	=&(\alpha \otimes \beta )(v,w)\cdot \gamma (z)
	\\
	=&(\alpha(v)\cdot \beta (w))\cdot \gamma(z)
	\\
	=&\alpha(v)\cdot [\beta(w)\gamma(z)]
	\\
	=&\alpha(v)\cdot [(\beta \otimes \gamma(w,z))]
	\\
	=&[\alpha\otimes(\beta\otimes\gamma)](v,w,z)
	\end{align*}
	$\therefore (\alpha\otimes\beta)\otimes\gamma=\alpha\otimes(\beta\otimes\gamma)$
	\\
	故多重线性函数的张量积服从分配率和结合律。
\end{itemize}





\noindent
\\
\\
{\textbf{1.20\ 证:}}
设\ $\{\xi_i\}$\ 为\ $V$\ 的基,\ $\{\eta_i\}$\ 为\ $W$\ 的基,记\ $A=\{V\otimes W|v\in V,w\in W \}$,\ 下证\ $A$\ 不构成线性空间:
\\
由于\ $dim(V)\geqslant 2,dim(W)\geqslant 2$.\ 所以分别可取\ $V,W$\ 的两组基:\ $\xi_1,\xi_2,\eta_1,\eta_2$,则
$$\xi_1\otimes \eta_1\in A,\ \xi_2\otimes\eta_2\in A,$$
若$$\xi_1\otimes\eta_1+\xi_2\otimes\eta_2\in A$$
则\ $\exists\  a^i\ (i=1,2,...,n)\ s.t.\ $
$$\xi_1\otimes \eta_1+\xi_2\otimes\eta_2=(a^i\xi_i)\otimes (b^j\eta_j )=a^ib^j\xi_i\otimes\eta_j$$
移项并合并后,由于\ $\{\xi_i\eta_j\}$\ 线性无关,所以有
\begin{displaymath}
a^ib^j=\left\lbrace \begin{array}{lll}
                      1 & i=j=1\ \text{以及}\ i=j=2\ \text{时};
                      \\
                      0 & \text{其余情况}.    
                     \end{array}
        \right.
\end{displaymath}
$a^1b^1=a^2b^2=1$,\ 从而\ $a^1,b^1,a^2,b^2\neq 0$,\ 但\ $a^1b^2=0$,\ 矛盾!
\\
$\therefore \xi_1\otimes\eta_1+\xi_2\otimes\eta_2\notin A$,\ $A$\ 不构成线性空间。


\noindent
\\
\\
{\textbf{1.21\ 证:}} 

\begin{itemize}
	\item [(1)] 
	由题知\;$f(\delta_i)=b_i^j\delta_i.$
	\\
	若另取基底\;$\{e_i\}$\; 且\;$e_i=a_i^j\delta_j$\; 从而\;$\delta_j=c_j^ie_i$\;(其中\;$(c_i^j)$\;为\;$(a_i^j)$\;的逆).
	\\
	则\;$f(e_i)=f(a_i^j\delta_j)=a_i^jf(\delta_j)=a_i^jb_j^k\delta_k=a_i^jb_j^kc_k^le_l$.
	\\
	即\;$f(e_i)=(a_i^jb_j^kc_k^l)e_l,$\;记\;$d_i^l=a_i^jb_j^kc_k^l.$
	\\
	$\therefore f$\;在基\;$\{e_i\}$\;下矩阵是\;$(d_i^l).$
	\\
	此时
	\begin{align*}
	B_3&=d_i^jd_k^id_j^k
	\\
	   &=a_i^s b_s^t c_t^j a_k^e b_e^f c_f^i a_j^g b_g^h c_h^k
	   \\
	   &=(a_i^s c_f^i) (a_k^e c_h^k) (a_j^g c_t^j) b_s^t b_e^f b_g^h
	   \\
	   &=(\delta_f^s b_e^f) (\delta_h^e b_g^h) (\delta_t^g b_s^t)
	   \\
	   &=b_e^s b_g^e b_s^g
	\end{align*}
	$\therefore B_3$\; 与基底的选取无关。 
	\item[(2)]
	令\;$F:V^{\star}\times V^{\star}\to \mathbb{R}$\;为$\; F(\delta^i,u)=\delta^i(f(u)).\;(\forall u\in V)$.\; 则易证$\;F\;$为\;(1,1)\;型张量,且
	\begin{align*}
	B_3&=b_i^j b_k^i b_j^k
	\\
	   &=F(\delta^j,\delta_i)\cdot F(\delta^i,\delta_k)\cdot F(\delta^k,\delta_j)
	   \\
	   &=F(\delta^j,\delta_i)\cdot [(C_1^2 F\otimes F)(\delta^i,\delta_j)]
	   \\
	   &=C_2^1[F\otimes C_1^2(F\otimes F)](\delta^j,\delta_j)
	   \\
	   &=C_1^1 \{C_2^1 [F\otimes C_1^2(F\otimes F)]\}.
	\end{align*}
\end{itemize}
\noindent
\\
{\textbf{1.22\ 解:}}\;令$\;F:V^{\star}\times V\to \mathbb{R}\quad F(v^i,v_ j)=v^i(v_j).$\,易证$\;F\;$为\;(1,1)\;型张量.\;
任取$\;V\;$的基底\;$\{\delta_i\}$,\;其对偶基底\;$\{\delta^i\}$,\;则\;$\{\delta_i\otimes \delta^j\}$\;为$\;F\;$的基底,且 \;$\delta^i(\delta_j)=\delta_j^i$.\;从而
$\; F(\delta^i,\delta_j)=\delta_i^j.$
$$F=F(\delta^i,\delta_j)\delta_i\otimes\delta^j=\delta_i^j\delta_i\otimes \delta^j=\sum_{i}\delta_i\otimes \delta^i$$
$\therefore \delta_i^j$\;为$\;F\;$的分量。
 
\noindent
\\
\\
{\textbf{1.23\ 解:}}\;若存在\;(0,2)\;型张量$\;F\;$满足题设条件,则任取$\;V\;$的一组基底$\;\{\delta_i\},\;F(\delta_i,\delta_j)=\delta_{ij}.\;$另任取一基底\;$\{e_i\}$\;,且\;$e_i=a_i^j\delta_j$,\;则
\begin{align*}
F(e_i,e_j)=\delta_{ij}\Leftrightarrow & F(a_i^k\delta_k,a_j^r\delta_r)=\delta_{ij}
\\
\Leftrightarrow &a_i^k a_j^r F(\delta_k,\delta_r)=\delta_{ij}
\\
\Leftrightarrow &a_i^k a_j^k=\delta_{ij} 
\\
\Leftrightarrow &(a_i^j)\; \text{为单位正交阵}
\end{align*}
但$\;a_i^j\;$未必为单位正交阵,矛盾!
\\
所以不存在。

\noindent
\\
\\
{\textbf{1.24\ 证:}}$\;\forall\; v_1,\cdots ,v_q \in V,\; (\sigma(f))(v_1,\cdots ,v_q)=f(v_{\sigma(1)},\cdots ,v_{\sigma(q)})=\alpha^1(v_{\sigma(1)})\cdots \alpha^q (v_{\sigma(q)}).$
\begin{align*}
\alpha^{\tau (1)}(v_1)\otimes \cdots \otimes \alpha^{\tau (q)}(v_1,\cdots ,v_q)&=\alpha ^{\tau(1)}\otimes \cdots \otimes \alpha ^{\tau(q)}(v_1,\cdots ,v_q)\;
\\
(\text{由}\; \tau =\sigma^{-1}\; \text{知}\; i=\sigma(\tau (i)))\quad     
&=\alpha^{\tau(1)}(v_{\sigma(\tau (1))})\cdots \alpha^{\tau(q)}(v_{\sigma(\tau (q))})
\\
&=\alpha^{1}(v_{\sigma(1)})\cdots \alpha^{q}(v_{\sigma(q)})
\end{align*}
$\therefore \sigma(f)=\alpha^{\tau(1)}\otimes\cdots\otimes \alpha^{\tau(q)}$

\noindent
\\
\\
{\textbf{1.25\ 证:}} 
\begin{itemize}
	\item [(1)] 
	   \begin{align*}
	   \sigma \;\text{是对称张量}\;&\Leftrightarrow \sigma(\xi)=\xi
	   \\
	                              &\Leftrightarrow \frac{1}{q!}\sum_{\sigma\in \varphi(q)}\sigma(\xi)=\xi
	                              \\
	                              &S_q(\xi) =\xi.                        
	   \end{align*}
    \item [(2)]
       \begin{align*}
         \sigma \;\text{是反对称张量}\; &\Leftrightarrow \sigma(\xi)=sign(\sigma)\xi
         \\                             
                                       &\Leftrightarrow sign(\sigma)\cdot \sigma(\xi)=\xi
                                       \\
                                       &\Leftrightarrow \sum_{\sigma\in \varphi(q)}sign(\sigma)\cdot \sigma(\xi)=\xi\cdot q!
                                       \\
                                       &\Leftrightarrow A_q(\xi)=\xi
       \end{align*}
	   
\end{itemize}


\noindent
\\
\\
{\textbf{1.26\ 证:}} $\forall\; \xi \in V_2^0,\{\delta^i\}\;\text{为}\;V\;\text{中的基,\;}$ 记  $\;\xi=\xi_{ij}\delta^i\delta^j\;$为二阶协变张量.\ 对于$\delta \stackrel{\Delta}{=}\delta^i\otimes \delta^j.$
\begin{align*}
S_2(\delta)&=\frac{1}{2}\cdot (\delta^i\otimes\delta^j+\delta^j\otimes\delta^i)\;(\text{由24题})
\\
A_2(\delta)&=\frac{1}{2}(\delta^i\otimes\delta^j-\delta^j\otimes\delta^i)
\\
\Rightarrow \delta&=\delta^i\otimes\delta^j=S_2(\delta)+A_2(\delta)
\\
\therefore \xi=&\xi_{ij}\delta^i\otimes\delta^j
\\
&=\delta_{ij}(S_2(\delta^i\otimes\delta^j)+A_2(\delta^i\otimes\delta^j))
\\
&=S_2(\xi)+A_2(\xi)
\end{align*} 
其中$\;S_2(\xi)\;$为对称张量,$A_2(\xi)\;$为反对称张量。

\noindent
\\
\\
{\textbf{1.27\ 证:}}取$\;V\;$中基底$\;\{\delta_i\},\;$由$\;\varphi\in V_2^0,\;$依题意知$,\;\forall\;\delta_i\delta_j\delta_k\;$有
\begin{align*}
\varphi(\delta_i,\delta_j,\delta_k)&=\varphi(\delta_j,\delta_i,\delta_k)
                                   \\
                                   &=-\varphi(\delta_j,\delta_k,\delta_i)
                                   \\
                                   &=-\varphi(\delta_k,\delta_j,\delta_i)
                                   \\
                                   &=\varphi(\delta_k,\delta_i,\delta_j)
                                   \\
                                   &=-\varphi(\delta_i,\delta_j,\delta_k)
         \\
         \Rightarrow &\varphi(\delta_i,\delta_j,\delta_k)=0.
         \\
         \therefore \varphi&=\varphi(\delta_i,\delta_j,\delta_k)\delta^i\otimes\delta^j\otimes\delta^k=0.
\end{align*}


\noindent
\\
\\
{\textbf{1.28\ 证:}}取$\;V\;$的基$\;\{\delta_i\},\;\because a(x,x)=0,\;\therefore a(\delta_i,\delta_j)=0\quad (i=j)\;$时.
\\
$\forall i\neq j\;$取$\;x=\delta_i+\delta_j\;$则
\begin{align*}
    0 &=a(x,x)
      \\
      &=a(\delta_i+\delta_j,\delta_i+\delta_j)
      \\
      &=a_{ii}+a_{ij}+a_{ji}+a_{jj}
      \\
      &=a_{ij}+a_{ji}.
\end{align*}
记$\;a=a_{ij}\delta^i\otimes \delta^j\;$有
\begin{align*}
S_2(a)&=\frac{1}{2}a_{ij}(\delta^i\otimes\delta^j+\delta^j\otimes\delta^i)
    \\
    &=\frac{1}{2}(a_{ij}\delta^i\otimes\delta^j+a_{ji}\delta^i\otimes \delta^j)
    \\
    &=\frac{1}{2}(a_{ij}+a_{ji})\delta^i\otimes\delta^j
    \\
    &=0.
\end{align*}

\noindent
\\
\\
{\textbf{1.29\ 证:}}由于$\;a\;$为任意对称张量,可取$\;a\;$使得$\;a^{ij}\neq 0(\forall\;i,j)\;$(\;例如,取$\;a=\sum_{i}\sum_{j}\delta_i\otimes\delta_j\;$), 则
$$0=a^{ij}b_{ij}=a^{ji}b_{ji}=a^{ij}b_{ji}$$
$\therefore a^{ij}(b^{ij}+b^{ji})=0,\;$此时$b_{ij}+b_{ji}=0\Rightarrow b_{ij}=-b{ji},$
\\
$\therefore b\;$为反对称张量。

\noindent
\\
\\
{\textbf{1.30\ 证:}}取\ $V$ 的基\ $\{\delta_i\}$ ,由\ $x$ 的任意性, $\forall\; i,j$\ 取\ $x=\delta_i+\delta_j \in V.$\ 由\ $f(x,x)=\tilde{f}(x,x)$\ 知
\begin{align*}
f(\delta_i+\delta_j,\delta_i+\delta_j)&=\tilde{f}(\delta_i+\delta_j,\delta_i+\delta_j)
\\
\Rightarrow f(\delta_i,\delta_i)+f(\delta_i,\delta_j)+f(\delta_j,\delta_i)+f(\delta_j,\delta_j)&=\tilde{f}(\delta_i,\delta_i)+\tilde{f}(\delta_i,\delta_j)+\tilde{f}(\delta_j,\delta_i)+\tilde{f}(\delta_j,\delta_j)
\\
\Rightarrow f(\delta_i,\delta_j)+f(\delta_j,\delta_i)&=\tilde{f}(\delta_i,\delta_j)+\tilde{f}(\delta_j,\delta_i)
\\
\Rightarrow 2f(\delta_i,\delta_j)&=2\tilde{f}(\delta_i,\delta_j)
\\
\Rightarrow f_{ij}&=\tilde{f}_{ij}
\end{align*}
记$\;f=f_{ij}\delta^i\otimes\delta^j \quad  \tilde{f}=\tilde{f}_{ij}\delta^i\otimes\delta^j,\;$则$\;f=\tilde{f}.$


\noindent
\\
\\
{\textbf{1.31\ 证:}}
原行列式$=\sum_{\sigma}\delta_{i_1,...,i_r}^{\sigma(i_1),...,\sigma(i_r)} \delta_{j_1}^{\sigma(i_1)}\cdots\delta_{j_r}^{\sigma(i_r)}=\delta_{i_1,...,i_r}^{j_1,...,j_r}.$


\noindent
\\
\\
{\textbf{1.32\ 证:}}
(1)\;记置换
$$
\pi=\left( 
         \begin{array}{l}
         i_1,\cdots , i_{r+s}
         \\
         j_1,\cdots, j_{r+s}
         \end{array}
    \right)
\quad
\pi_1=\left(
          \begin{array}{c}
          k_1,\cdots,k_r
          \\
          j_1,\cdots,j_r
          \end{array}
      \right)
\quad
\pi_2=\left(
          \begin{array}{c}
          i_1,\cdots,i_r,i_{r+1},\cdots,i_{r+s}
          \\
          k_1,\cdots,k_r,j_{r+1},\cdots,j_{r+s}
          \end{array}
      \right)
$$
则$\;\pi=\pi_1\pi_2,\;$从而$\;sgn(\pi)=sgn(\pi_1)\cdot sgn(\pi_2).$
\\
即,取定某排列$\;(l_1,\cdots,l_r),\;$有
$$
\delta_{j_1,\cdots,j_{r+s}}^{i_1,\cdots,i_{r+s}}=\delta_{j_1,\cdots,j_{r}}^{l_1,\cdots,l_r}\cdot\delta_{l_1,\cdots,l_r,j_{r+1},\cdots,j_{r+s}}^{i_1,\cdots,i_{r+s}}\quad(\text{注:此时不对上下指标求和})
$$
而$\;(l_1,\cdots,l_r)\;$的排列共有$\;r!\;$种取法.
$$\therefore \delta_{j_1,\cdots,j_{r+s}}^{i_1,\cdots,i_{r+s}}=\frac{1}{r!}\sum_{k_1,\cdots,k_r}\delta_{j_1,\cdots,j_r}^{k_1,\cdots,k_r}\delta_{k_1,\cdots,k_r,j_{r+1},\cdots,j_{r+s}}^{i_1,\cdots,i_{r+s}}.$$

(2)$$\delta_{i_1,\cdots,i_r}^{i_1,\cdots,i_r}=A_n^r=\frac{n!}{(n-r)!}.$$

\noindent
\\
\\
{\textbf{1.33\ 证:}}记$\;a=(\alpha_1,...,\alpha_n),\;$其中$\;\alpha_i=(a_i^1,...,a_i^n)^T$,则
\begin{align*}
\det a&=\det(\alpha_1,...,\alpha_n)
      \\
      &=\frac{1}{n!}\delta^{i_1,...,i_n}_{1,...,n} \det(\alpha_{i_1},...,\alpha_{i_n})
      \\
      &=\frac{1}{n!} \delta^{i_1,...,i_n}_{1,...,n} \delta_{j_1,...,j_n}^{1,...,n} a_{i_1}^{j_1}\cdots a_{j_n}^{i_n}
      \\
      &=\frac{1}{n!}\delta_{j_1,...,j_n}^{i_1,...,i_n}a_{i_1}^{j_1}\cdots a_{j_n}^{i_n}.
\end{align*}


\noindent
\\
\\
{\textbf{1.34\ 证:}}